For two decades, a comprehensive, three-dimensional global atmospheric
 general circulation model (GCM) is being provided by the National
 Center for Atmospheric Research (NCAR, Climate and Global Dynamics Division)
 to university and other scientists for use in analysing and understanding
 the global climate. Designed as a Community Climate Model (CCM)
 it has been continuously developed since. Other centres have also
 constructed comprehensive climate models of similarly high complexity,
 mostly for their research interests.

As the complexity of general circulation models has been and still
 is growing considerably, it is not surprising that, for both
 education and research, models simpler than those comprehensive
 GCMs at the cutting edge of the development, are becoming more
 and more attractive. These medium complexity models do not simply
 enhance the climate model hierarchy. They support understanding
 atmospheric or climate phenomena by simplifying the system
 gradually to reveal the key mechanisms. They also provide an
 ideal tool kit for students to be educated and to teach themselves,
 gaining practice in model building or modeling. Our aim is to
 provide such a model of intermediate complexity for the university
 environment: the PlanetSimulator. It can be used for training
 the next GCM developers, to support scientists to understand
 climate processes, and to do fundamental research.

From PUMA to PlanetSimulator: Dynamical core and physical processes
 comprise a general circulation model (GCM) of planetary atmospheres.
 Stand-alone, the dynamical core is a simplified general circulation
 model like our Portable University Model of the Atmosphere or PUMA.
 Still, linear processes are introduced to run it, like Newtonian
 cooling and Rayleigh friction, which parameterise diabatic heating
 and planetary boundary layers. Though simple, PUMA has been enjoying
 a wide spectrum of applications and initiating collaborations in
 fundamental research, atmospheric dynamics and education alike.
 Specific applications, for example, are tests and consequences
 of the maximum entropy production principle, synchronisation and
 spatio-temporal coherence resonance, large scale dynamics of the
 atmospheres on Earth, Mars and Titan. Based on this experience we
 combined the leitmotifs behind PUMA and the Community Model, to
 applying, building, and coding a 'PlanetSimulator'.

Applying the PlanetSimulator in a university environment has two
 aspects: First, the code must be open and freely available as
 the software required to run it; it must be user friendly,
 inexpensive and equipped with a graphical user interface.
 Secondly, it should be suitable for teaching project studies
 in classes or lab, where students practice general circulation
 modelling, in contrast to technicians running a comprehensive GCM;
 that is, science versus engineering.

Building the PlanetSimulator includes, besides an atmospheric
 GCM of medium complexity, other compartments of the climate
 system, for example, an ocean with sea ice, a land surface with
 biosphere. Here these other compartments are reduced to linear
 systems. That is, not unlike PUMA as a dynamical core with
 linear physics, the PlanetSimulator consists of a GCM with,
 for example, a linear ocean/sea-ice module formulated in
 terms of a mixed layer energy balance. The soil/biosphere
 module is introduced analoguously. Thus, working the PlanetSimulator
 is like testing the performance of an atmospheric or oceanic GCM
 interacting with various linear processes, which parameterise
 the variability of the subsystems in terms of their energy
 (and mass) balances.

Coding the PlanetSimulator requires that it is portable to many
 platforms ranging from  personal computers over workstations
 to mainframes; massive parallel computers and clusters of
 networked machines are also supported. The system is scalable
 with regard to vertical and horizontal resolutions, provides
 experiment dependent model configurations, and it has a transparent
  and rich documented code.

Acknowledgement: The development of the Planet Simulator
 was generously granted by the
 German Federal Ministry for Education and Research (BMBF)
 during the years 2000 - 2003.

